% !TeX spellcheck = en_GB 

\chapter{Exercises}
\section{Sheet 0 - Questions in the lecture slides}

\subsection{Q1 - lecture 02} 
Assume you have an RGB image of resolution $1024\times 768$ with $256$ intensity levels per color channel. How many bytes of memory does such an image require when loaded into a computer?
\paragraph{Solution:} 
To store $256=2^8$ intensity levels exactly eight bits (one byte) is required. Thus to store $1024\times 768 = 786,432$ pixels (intensity levels), \num[group-separator={,}]{786432} bytes are required.

\subsection{Q2 - lecture 03} 
The \textbf{divergence} of a continuous multivariate function $f(x_1,x_2,\dots,x_n)$ is defined as 
\begin{equation}
	\Delta f = \nabla^2 f = \angles{\nabla}{\nabla f} = \sum_{i=1}^{n} \frac{\partial^2 f}{\partial x_i^2}
\end{equation}
where $\Delta$ is called the \textbf{Laplace operator}.

Write down an expression for the divergence of a discrete function $g[x,y]$.
\paragraph{Solution:} 

\subsection{Q3 - lecture 03} 
Assume a discrete uni-variate function $g[x]$. What does the following operator do?
\begin{equation}
	\text{AM}g[x] = \frac{1}{3}\big( g[x-1] + g[x] + g[x+1] \big)
\end{equation}
\paragraph{Solution:} 

\subsection{Q4 - lecture 03} 
Assume a discrete uni-variate function $g[x]$. What does the following operator do?
\begin{equation}
	\text{WM}g[x] = \frac{1}{4}\big( g[x-1] + 2g[x] + g[x+1] \big)
\end{equation}
\paragraph{Solution:} 

\newpage
\section{Sheet 1}

\subsection{Assignment 1a - Bias of an estimator} 
\dots
\paragraph{Solution:} 
